%%%%%%%%%%%%%%%%%%%%%%%%%%%%%%%%%%%%%%%%%%%%%%%%%%%%%%%%%%%%%%%%%%%%%%
% writeLaTeX Example: A quick guide to LaTeX
%
% Source: Dave Richeson (divisbyzero.com), Dickinson College
% 
% A one-size-fits-all LaTeX cheat sheet. Kept to two pages, so it 
% can be printed (double-sided) on one piece of paper
% 
% Feel free to distribute this example, but please keep the referral
% to divisbyzero.com
% 
%%%%%%%%%%%%%%%%%%%%%%%%%%%%%%%%%%%%%%%%%%%%%%%%%%%%%%%%%%%%%%%%%%%%%%
% How to use writeLaTeX: 
%
% You edit the source code here on the left, and the preview on the
% right shows you the result within a few seconds.
%
% Bookmark this page and share the URL with your co-authors. They can
% edit at the same time!
%
% You can upload figures, bibliographies, custom classes and
% styles using the files menu.
%
% If you're new to LaTeX, the wikibook is a great place to start:
% http://en.wikibooks.org/wiki/LaTeX
%
%%%%%%%%%%%%%%%%%%%%%%%%%%%%%%%%%%%%%%%%%%%%%%%%%%%%%%%%%%%%%%%%%%%%%%

\documentclass[dvipdfmx,a4paper,10pt,landscape]{article}

\usepackage{amsmath}
\usepackage{stix2}
\usepackage{siunitx}

\usepackage{multicol,multirow}
\usepackage{ifthen}
\usepackage[landscape]{geometry}
\usepackage[colorlinks=true,citecolor=blue,linkcolor=blue]{hyperref}

\ifthenelse{\lengthtest { \paperwidth = 11in}}
    { \geometry{top=.5in,left=.5in,right=.5in,bottom=.5in} }
	{\ifthenelse{ \lengthtest{ \paperwidth = 297mm}}
		{\geometry{top=1cm,left=1cm,right=1cm,bottom=1cm} }
		{\geometry{top=1cm,left=1cm,right=1cm,bottom=1cm} }
	}
\pagestyle{empty}
\makeatletter
\renewcommand{\section}{\@startsection{section}{1}{0mm}%
                                {-1ex plus -.5ex minus -.2ex}%
                                {0.5ex plus .2ex}%x
                                {\normalfont\large\bfseries}}
\renewcommand{\subsection}{\@startsection{subsection}{2}{0mm}%
                                {-1explus -.5ex minus -.2ex}%
                                {0.5ex plus .2ex}%
                                {\normalfont\normalsize\bfseries}}
\renewcommand{\subsubsection}{\@startsection{subsubsection}{3}{0mm}%
                                {-1ex plus -.5ex minus -.2ex}%
                                {1ex plus .2ex}%
                                {\normalfont\small\bfseries}}
\makeatother
\setcounter{secnumdepth}{0}
\setlength{\parindent}{0pt}
\setlength{\parskip}{0pt plus 0.5ex}

\DeclareMathOperator{\sinc}{sinc}
\DeclareMathOperator{\rect}{rect}
% -----------------------------------------------------------------------

\begin{document}

\raggedright
\footnotesize

\begin{center}
    \Large{\textbf{EE4704 Cheat Sheet}} \\
\end{center}
\begin{multicols}{3}
    \setlength{\premulticols}{1pt}
    \setlength{\postmulticols}{1pt}
    \setlength{\multicolsep}{1pt}
    \setlength{\columnsep}{2pt}

    \section{Introduction}

    \subsection{Network Delays and Losses}
    $d_{\text{nodal}}=d_{\text{proc}}+d_{\text{queue}}+d_{\text{trans}}+d_{\text{pop}}$\\
    $d_{\text{trans}}=\frac{L\text{(bits)}}{R(\text{bits/sec})}$ ($L$: length of packet, $R$: transmission rate)\\
    $d_{\text{queue}}=d_{\text{trans}}\times l_{\text{queue}}$, ($l_{\text{queue}}$: average length of queue) \\
    $d_{\text{prop}}=\frac{D}{s}$ ($D$: distance between nodes, $s$: propagation speed)\\

    \subsection{Packet Switching vs Circuit Switching}
    \begin{itemize}
        \item great for bursty data
        \item simpler, no call setup
        \item excessive congestion possible: packet delay and loss
        \item protocols needed for reliable data transfer, congestion control
    \end{itemize}

    \subsection{Throughput}
    throughput: rate (bits/time unit) at which bits transferred between
    sender/receiver

    \subsection{Internet Protocol Stack}
    \begin{itemize}
        \item \textbf{application}: supporting network applications
              FTP, SMTP, HTTP
        \item presentation: allow applications to
              interpret meaning of data, e.g., encryption,
              compression, machine-specific
              conventions
        \item session: synchronization, checkpointing,
              recovery of data exchange
        \item \textbf{transport}: process-process data transfer
              TCP, UDP
        \item \textbf{network}: routing of datagrams from source to
              destination
              IP, routing protocols
        \item \textbf{link}: data transfer between neighboring
              network elements
              Ethernet, 802.111 (WiFi), PPP
        \item \textbf{physical}: bits ``on the wire''
    \end{itemize}
    Presentation and session from ISO/OSI must be implemented in application \textit{if needed}

    \section{Mobile IP}
    Pros:
    \begin{enumerate}
        \item no need to propagate routing updates again
        \item no need to modify routers
        \item does not create additional entries in routing table
        \item preserves location privacy
    \end{enumerate}
    Cons:
    \begin{enumerate}
        \item inefficiency of triangle routing
        \item packet header overhead of encapsulation
        \item home agent is single point of failure
        \item need to allocate an adress in foreign network
    \end{enumerate}

    \section{Ethernet Frame}
    Source IP\\
    Destination IP\\
    Source MAC\\
    Destination MAC

    If not on same LAN, destination MAC is of router

    \section{Security}
    Nonce is used once in a lifetime, used to prevent playback attacks.

    A certificate authority certifies that a public key belongs to a particular
    person or entity. It is useful to prevent adversaries from sending information or
    authenticating while pretending to be someone else.\\

    \section{Fourier Transform}
    Delta function:
    1D:
    \begin{equation*}
        \delta(x)=
        \begin{cases}
            0      & x \neq 0 \\
            \infty & x = 0
        \end{cases}
    \end{equation*}
    $f(x)\times\delta(x-a)=f(a)\delta(x-a)$ \\
    $\int_{-\infty}^{\infty}f(x)\delta(x-a)dx=f(a)$
    2D:
    \begin{equation*}
        \delta(x,y)=
        \begin{cases}
            0      & x,y \neq 0 \\
            \infty & x,y = 0
        \end{cases}
    \end{equation*}
    $    f(x,y)\delta(x-a,y-b)=f(a,b)\delta(x-a,y-b)$ \\
    $\iint_{-\infty}^{+\infty}f(x,y)\delta(x-a,y-b)dxdy=f(a,b)
    $
    Rectangle function:
    \begin{equation*}
        \rect(x)   =
        \begin{cases}
            1 & |x|\leq\frac{1}{2} \\
            0 & \text{otherwise}
        \end{cases}
    \end{equation*}
    \begin{equation*}
        \rect(x,y) =
        \begin{cases}
            1 & |x|,|y|\leq\frac{1}{2} \\
            0 & \text{otherwise}
        \end{cases}
    \end{equation*}
    Sinc function:
    $
        \sinc(x)    =\frac{sin(\pi x)}{\pi x}     $                          \\
    $\sinc(x,y)  = \sinc(x)\sinc(y)
    $

    \subsection{2D Fourier Transform}
    $
        F(u,v)=\iint_{-\infty}^{+\infty}f(x,y)\exp[-j2\pi(ux+vy)]dxdy
    $\\
    Inverse:
    $
        f(x,y)=\iint_{-\infty}^{+\infty}F(u,v)\exp[j2\pi(ux+vy)]dudv
    $\\
    Note
    $
        \exp[j2\pi(ux+vy)]   =\cos[2\pi(ux+vy)]+j\sin[2\pi(ux+vy)]    $\\
    $  \exp[-j2\pi(ux+vy)]  =\cos[2\pi(ux+vy)]-j\sin[2\pi(ux+vy)]
    $\\
    In general, the Fourier transform is complex:
    $
        F(u,v)\equiv F_R(u,v)+jF_I(u,v)\equiv|F(u,v)|\exp[j\phi(u,v)]
    $
    with real part $F_R(u,v)$ and an imaginary part $F_I(u,v)$:
    \begin{center}
        \begin{tabular} {lll}
            Fourier spectrum: & $|F(u,v)|$  & $=[F_R^2(u,v)+F_I^2(u,v)]^{1/2}$ \\
            Phase spectrum:   & $\phi(u,v)$ & $=\tan^{-1}[F_I(u,v)/F_R(u,v)]$  \\
            Power spectrum:   & $P(u,v)$    & $=|F(u,v)|^2=R^2(u,v)+I^2(u,v)$
        \end{tabular}
    \end{center}
    Useful transform pairs:
    \begin{center}
        \begin{tabular}{ll}
            \hline
            Signal                     & Fourier transform                          \\
            \hline
            1                          & $\delta(u,v)$                              \\
            $\delta(x,y)$              & $1$                                        \\
            $\rect(x,y)$               & $\sinc(u,v)$                               \\
            $\sinc(x,y)$               & $\rect(u,v)$                               \\
            $e^{j2\pi(x+y)}$           & $\delta(u-1,v-1)$                          \\
            $e^{-(x^2+y^2)/2\sigma^2}$ & $2\pi\sigma^2e^{-2\pi^2\sigma^2(u^2+v^2)}$
        \end{tabular}
    \end{center}

    \subsection{Properties of 2D Continuous Transformation}
    Linearity:
    $
        \mathcal{F}\{af_1(x,y)+bf_2(x,y)\}=aF_1(u,v)+bF_2(u,v)
    $
    where $a$ and $b$ are constants.\\
    Conjugate Symmetry:
    $
        \mathcal{F}\{f^*(x,y)\}=F^*(-u,-v) $\\
    $|F(u,v)|=|F(-u,-v)|
    $
    where $^*$ denotes the complex conjugation of a variable,
    and $|F(u,v)|$ is symmetrical about the origin.\\
    Translation:
    $
        \mathcal{F}\{f(x-a,y-b)\}=F(u,v)\exp[-j2\pi(au+bv)]$ \\
    $  |\mathcal{F}\{f(x-a,y-b)\}|=|\mathcal{F}\{f(x,y)\}|
    $
    a spatial shift does not affect the magnitude of the transform.\\
    Scaling:
    $
        \mathcal{F}\{f(ax,by)\}=\frac{1}{|ab|}F(u/a,v/b)
    $\\
    Parseval's Theorem:
    $
        \iint_{-\infty}^{+\infty}|f(x,y)|^2dxdy=\iint_{-\infty}^{+\infty}|F(u,v)|^2 dudv
    $
    Convolution:
    $
        \mathcal{F}\{f(x,y)*h(x,y)\}=F(u,v)H(u,v)$\\
    $\mathcal{F}\{f(x,y)h(x,y)\}=F(u,v)*H(u,v)
    $\\
    Rotation:
    $
        x=r\cos\theta\quad y=r\sin\theta\quad u=\omega\cos\phi\quad v=\omega\sin\phi
    $
    then $f(x,y)$ and $F(u,v)$ become $f(r,\theta)$ and $F(\omega,\phi)$ and
    $
        f(r,\theta+\theta_0) \iff F(\omega,\phi+\theta_0)
    $
    \subsection{Discrete Fourier Transform}
    The discrete Fourier transform (DFT) pair is given by
    $F(u)  = \frac{1}{N}\sum_{x=0}^{N-1}f(x)\exp[-j2\pi ux/N];\quad u=0,1,2,\ldots,N-1$ \\
    $f(x)  = \sum_{u=0}^{N-1}F(u)\exp[j2\pi ux/N];\quad x=0,1,2,\ldots,N-1    $
    In the 2D case, the DFT pair is
    $  F(u,v)  = \frac{1}{MN}\sum_{x=0}^{M-1}\sum_{y=0}^{N-1}f(x,y)\exp[-j2\pi (ux/M+vy/N)];$ \\
    $u=0,1,2,\ldots,M-1,\ v=0,1,2,\ldots,N-1        $    \\
    $f(x,y)  = \sum_{u=0}^{M-1}\sum_{v=0}^{N-1}F(u,v)\exp[j2\pi (ux/M+vy/N)];   $    \\
    $x=0,1,2,\ldots,M-1, y=0,1,2,\ldots,N-1$

    Time Complexity:
    Direct DFT:  $M^2N^2$     ,
    FFT:         $MN\log_2(MN)$
    \subsection{Some Properties of the 2D DFT}
    Separability:
    $
        F(u,v) = \frac{1}{M}\sum_{x=0}^{N-1}F_r(x,v)\exp[-j2\pi ux/M]$ \\
    $F_r(x,v)=\frac{1}{N}\sum_{y=0}^{N-1}f(x,y)\exp[-j2\pi vy/N]
    $\\
    Average Value:
    $
        \bar{f}(x,y)=\frac{1}{MN}\sum_{x=0}^{M-1}\sum_{y=0}^{N-1}f(x,y) $\\
    $\bar{f}(x,y)=F(0,0)$\\
    Translation:
    Same as continuous.\\
    Conjugate Symmetry and Periodicity:
    $
        F(u)=F(u+kN)\quad k=0,\pm1,\pm2,\ldots $\\
    $f(x)=f(x+kN)\quad k=0,\pm1,\pm2,\ldots $\\
    $F(u) = F^*(-u) = F^*(-u+N)
    $\\
    Centering:
    $
        f'(x,y)=f(x,y)(-1)^{x+y}
    $

    \subsection{Image Sampling}
    $
        F_s(u,v)=\frac{1}{\Delta x \Delta y}\sum_{m=-\infty}^{\infty}\sum_{n=-\infty}^{\infty}F(u-mu_s,v-nv_s) $\\
    $ F_s(u)=\sum_{m=-\infty}^{\infty}F(u-mu_s)
    $
    Nyquist Sampling Theorem:
    $
        u_s\geq2U
    $
    otherwise, there is aliasing, leading to artifacts in sampled signal.

    \section{Noise Reduction}
    \subsection{Mean Filtering}
    Variations in the mask:
    $
        \frac{1}{9}
        \begin{bmatrix}
            1 & 1 & 1 \\
            1 & 1 & 1 \\
            1 & 1 & 1
        \end{bmatrix}
        \quad
        \frac{1}{10}
        \begin{bmatrix}
            1 & 1 & 1 \\
            1 & 2 & 1 \\
            1 & 1 & 1
        \end{bmatrix}
        \quad
        \frac{1}{16}
        \begin{bmatrix}
            1 & 2 & 1 \\
            2 & 4 & 2 \\
            1 & 2 & 1
        \end{bmatrix}
    $
    Causes blurring, is equivalent to low-pass filtering.
    Gaussian mask:
    $
        h(x,y)=\exp\left[-\frac{(x^2+y^2)}{2\sigma^2}\right]
    $
    \subsection{Median Filtering}
    Requires $3(N^2-1)/8$ comparisons with bubble sort to find $(N+1)/2$ largest value.
    \subsubsection{Properties}
    Good at removing outlier noise;
    Different window shapes and sizes may be used;
    The shape chosen for the window may affect the processing results;
    Reduces the variance of the intensities in the image;
    No new gray values are generated;
    In general, the median filter tends to preserve edges while removing noise effectively;

    \subsection{Spatial Filtering From Order Statistics}
    Midpoint Filter:
    $
        \text{output}=\frac{z_1+z_N}{2}
    $
    should not be used with images that contain outlier noise such as salt-and-pepper noise.\\
    Maximum and minimum filters:
    Minimum:
    $
        \text{output}=z_1
    $
    If an image contains only salt noise, then the minimum filter removes this noise.
    Maximum:
    $
        \text{output}=z_N
    $
    It is effective in smoothing an image containing only pepper noise. Both are biased.\\
    Alpha-Trimmed Mean Filter:
    Mixture of the mean and median filters, reasonably well for Gaussian and outlier.
    $
        \text{output}=\frac{1}{N-2p}\sum_{i=p+1}{N-p}z_i
    $
    where
    $
        p=0,1,2,3,\ldots,\frac{N-1}{2}\quad\text{N odd}
    $,
    $p=0$: mean, other median.\\
    Both outlier and Gaussian: $p\neq0$ removes outlier in calculation of mean, less blurring than mean filter.\\
    \subsection{Image Averaging}
    Consider noise uncorrelated over time and has mean value equal to zero.
    $
        g(x,y)=f(x,y)+\eta(x,y)$\\
    $\bar{g}(x,y)=f(x,y)+\frac{1}{K}\sum_{t=1}^K \eta_t(x,y)$\\
    $E\{\bar{g}(x,y)\}=f(x,y),\quad K\to\infty$

    \section{Image Enhancement}
    Contrast Stretching:
    \begin{equation*}
        s_k=\begin{cases}
            ar_k                & 0\leq r_k < r_a    \\
            \beta(r_k-r_a)+s_a  & r_a \leq r_k < r_b \\
            \gamma(r_k-r_b)+s_b & r_b \leq r_k < L-1
        \end{cases}
    \end{equation*}
    Digital Negative:
    $
        s_k=(L-1)-r_k
    $
    Intensity Level Slicing\\
    Log Transformation\\
    Histogram Equalization\\
    Histogram Specification
    $
        z=G^{-1}(s)=G^{-1}[T(r)]
    $
    \subsection{Image Sharpening}
    \subsubsection{Butterworth Filter}
    $
        H(\omega)=\frac{1}{1+(\omega/\omega_0)^{2n}}
    $
    No sharp discontinuity, less ringing. Swap $\omega$ and $\omega_0$ for high-pass.

    \subsection{Gaussian Filtering}
    $
        H(u,v)=e^{-\omega^2/2\sigma^2}=e^{-\ln 2(\omega/\omega_0)^2}\quad\text{low}
    $

    \section{Edge Detection}
    \subsection{Gradient Operators}
    $
        \mathbf{G}(x,y)=
        \begin{bmatrix}
            \partial f(x,y)/\partial x \\
            \partial f(x,y)/\partial y
        \end{bmatrix}=
        \begin{bmatrix}
            G_x \\
            G_y
        \end{bmatrix}
    $
    corrsponding to mask operations
    \begin{tiny}
        $
            \begin{bmatrix}
                -1 & 1
            \end{bmatrix}
            \quad\text{and}\quad
            \begin{bmatrix}
                1 \\
                -1
            \end{bmatrix}$\\
        $\begin{bmatrix}
                -1 & 0 & 1
            \end{bmatrix}
            \quad\text{and}\quad
            \begin{bmatrix}
                1 \\
                0 \\
                -1
            \end{bmatrix}\quad\text{(centered)} $\\
        $D_{P_x}=\begin{bmatrix}
                -1 & 0 & 1 \\
                -1 & 0 & 1 \\
                -1 & 0 & 1
            \end{bmatrix}
            \quad\text{and}\quad
            D_{P_y}=\begin{bmatrix}
                1  & 1  & 1  \\
                0  & 0  & 0  \\
                -1 & -1 & -1
            \end{bmatrix}\quad\text{(Prewitt)}$ \\
        $D_{S_x}=
            \begin{bmatrix}
                -1 & 0 & 1 \\
                -2 & 0 & 2 \\
                -1 & 0 & 1
            \end{bmatrix}
            \quad\text{and}\quad
            D_{S_y}=
            \begin{bmatrix}
                1  & 2  & 1  \\
                0  & 0  & 0  \\
                -1 & -2 & -1
            \end{bmatrix}\quad\text{(Sobel)}$ \\
        $D_+=
            \begin{bmatrix}
                0  & 1 \\
                -1 & 0
            \end{bmatrix}
            \quad\text{and}\quad
            D_-=\begin{bmatrix}
                1 & 0  \\
                0 & -1
            \end{bmatrix}\quad\text{(Roberts)}$ \\
    \end{tiny}
    Magnitude of the gradient is
    $
        |\mathbf{G}(x,y)|  =[G_x^2+G_y^2]^{1/2}  \approx|G_x|+|G_y|
    $
    Direction of gradient (direction of max rate of change) w.r.t. x axis
    $
        \theta(x,y)=\tan^{-1}(G_y/G_x)
    $
    \subsection{Laplacian Operator}
    $
        L[f(x,y)]=\frac{\partial^2f}{\partial x^2}+\frac{\partial^2f}{\partial y^2}
    $
    A Laplacian mask is
    $
        \begin{bmatrix}
            0 & 1  & 0 \\
            1 & -4 & 1 \\
            0 & 1  & 0
        \end{bmatrix}
    $
    \subsection{Derivative Theorem of Convolution}
    $
        \frac{d}{dx}f(f*g)=f*\frac{d}{dx}g
    $

    \section{Segmentation}
    \subsection{Contour Tracking}
    \begin{enumerate}
        \item Search top-bottom, left-right for edge pixel, $N$ (direction w.r.t preceding) set to $1$, directions:
              \begin{tiny}
                  $\begin{bmatrix}
                          6 & 7 & 0 \\
                          5 &   & 1 \\
                          4 & 3 & 2
                      \end{bmatrix}$
              \end{tiny}
        \item $M=(N-2)\ \text{modulo}\ 8$
        \item Update $N=M$ if found, otherwise $M=(M+1)\ \text{modulo}\ 8$
    \end{enumerate}
    \subsection{Local Analysis}
    Magnitude criterion, direction criterion, curvature of segment, closeness to known contour
    \subsection{Global Analysis -- Hough Transform}
    Slope intercept:
    $
        y_i=ax_i+b\to b=-x_ia+y_i
    $\\
    Normal:
    $
        x\cos\theta+y\sin\theta=\rho,\quad \qty{-90}{\degree}<\theta\leq\qty{90}{\degree}$\\
    $\rho\equiv\sqrt{x_i^2+y_i^2}\cos(\theta-\tan^{-1}(y_i/x_i))$\\
    $ \rho_i=\sqrt{x_i^2+y_i^2},\quad \theta_i=\tan^{-1}(y_i/x_i)$\\
    Extensions of the Hough Transform:
    $
        (x_i-a_1)^2+(y_i-a_2)^2=r^2\quad\text{(circle, 3D)}$\\
    $  \frac{(x-x_0)^2}{a^2}+\frac{(y-y_0)^2}{b^2}=1\quad\text{(ellipse, 5D, need orientation $\phi$)}
    $
    \subsection{Adaptive Theresholding}
    Niblack thresholding, Sauvola thresholding, and Bradley thresholding. Find valley.\\
    Histogram Smoothing:
    Replace bin value by average of its neighbors.\\
    Intermeans Algorithm:
    Initial $T$: mean or median of image, partition using T, compute $\mu_1$, $\mu_2$, $T=\frac{1}{2}(\mu_1+\mu_2)$, repeat.
    \subsection{Labeling of Connected Components (Binary)}
    Scan until pixel $p$ for which $V=\{1\}$, examin 4 neighbors already encountered, assign any most popular label or new if none, repeat.\\
    Note equivalence if multiple most popular.
    \subsection{Region-Oriented Methods (Grayscale or RGB)}
    Region growing: assign unassigned neighbors of region that satisfy uniformity predicate
    Problems: selection of initial seeds, suitable properties, formulation of stopping rule, incomplete segmentation
    Region Splitting/Merging: split into disjoint quadrants if region does not satisfy predicate, merge adjacent regions that satisfy same predicate

    \section{Representation and Description}
    Chiain code
    \begin{tiny}
        $\begin{bmatrix}
                3 & 2 & 1 \\
                4 &   & 0 \\
                5 & 6 & 7
            \end{bmatrix}$
    \end{tiny}:
    Traverse in clockwise direction for every pair of pixels; resample buondary by selecting larger grid spacing; divide into segments of equal length.
    Signatures\\
    Distance-angle function:\\
    Centroid estimate from boundary $B=\{(x_i,y_i),\quad i=1,2,\ldots,N\}$:\\
    $\bar{x}=\frac{1}{N}\sum_i x_i$, $\bar{y}=\frac{1}{N}\sum_i y_i$\\
    Slope-density function ($\psi$--$s$):\\
    Tangential orientation $\psi\ (0\leq \psi \leq 2\pi)$ w.r.t. the horizontal, as a function of boundary distance $s$.
    Shape Descriptors
    Area;
    Perimeter: number of boundary pixels, multiply diagonal boundary segments by $\sqrt{2}$;
    Compactness: $\gamma=\frac{\text{(perimeter)}^2}{4\pi\times(\text{area})}$;
    Radial distances;
    Boudning box;
    Eccentricity: $\varepsilon=\frac{\text{length of major axis}}{\text{length of minor axis}}\ \text{or}\ \frac{\text{maximum radial distance}}{\text{minimum radial distance}}$;
    Axis of symmetry;\\
    Topological Descriptors\\
    Unaffected by any deformation, as long as no tearing or joining\\
    Euler number: $E=C-H$, connected minus hole, RST invariant
    \subsection{Image moments}
    $m_{pq}=\sum_x \sum_y x^p y^q f(x,y)$, centroids: $\bar{x}=\frac{m_{10}}{m_{00}}\quad \bar{y}=\frac{m_{01}}{m_{00}}$
    Properties\\
    Scaling: ratio of two moments with same $p+q$ invariant\\
    Symmetry: $m_{pq}=0$ for $p$ odd -- symmetric about $y$-axis,
    $m_{pq}=0$ for $q$ odd -- symmetric about $x$-axis,
    $m_{pq}=0$ for $p+q$ odd -- symmetric about origin
    \subsubsection{Central moment}
    Translation invariant: $\mu_{pq}=\iint(x-\bar{x})^p(y-\bar{y})^qf(x,y)dxdy$\\
    $\mu_{00}=m_{00}$, $\mu_{10}=0$, $\mu_{01}=0$, $\mu_{20}=\sum\sum(x-\bar{x}^2)f(x,y)$,
    $\mu_{02}=\sum\sum(y-\bar{y}^2)f(x,y)$

    \subsubsection{Normalized Central Moment}
    Scale and translation invariant: $\eta_{pq}=\frac{\mu_{pq}}{\mu_{00}^\gamma}$, where $\gamma=\frac{p+q}{2}+1\quad \text{for } p+q=2,3,\ldots$
    \subsubsection{Invariant moment}
    RST invariant: $\phi_1=\eta_{20}+\eta_{02}$, digital not strictly true for RT
    \subsection{Texture}
    Relative smoothness: $R(z)=1-\frac{1}{1+\sigma^2(z)}$, normalized: $R_n(z)=1-\frac{1}{1+[\sigma(z)/(L-1)]^2}$
    $\mu_3$ is skewness and $\mu_4$ is kurtosis.
    \subsection{Second-order Gray Level Statistics}
    Specify displacement $\delta(\Delta x, \Delta y)$, matrix $\mathbf{A}$ element $a_{i,j}$ is number of times gray level pair occur for displacement\\
    Gray-level co-occurence matrix (GLCM): $\mathbf{C}=\frac{1}{n_p}\mathbf{A}$, estimate of joint probability\\
    Maximum probability: $D_m=\max(c_{ij})$, element-difference moment of order $k$: $D_{ed,k}\sum_i\sum_j(i-j)^kc_{ij}$,
    inverse element-difference moment of order $k$: $D_{ied,k}=\sum_i\sum_j c_{ij}$,
    entropy: $D_e=-\sum_i\sum_j c_{ij} \log_2 c_{ij}$, uniformity: $D_u=\sum_i\sum_j c_{ij}^2$

    \section{Morphological Processing}
    Dilation: $A \oplus B=\{x\vert (\hat{B})_x\cap A \neq \emptyset\}$ \\
    Erosion: $A \ominus B =\{x\vert (B)_x \subseteq A\}$ \\
    Opening: $A \circ B = (A \ominus B) \oplus B=\bigcup\{(B)_x\vert(B)_x\subseteq A\}$ \\
    Closing: $A \bullet B = (A \oplus B) \ominus B = \text{ complement of } \bigcup\{(\hat{B})_x\vert(\hat{B})_x\subseteq A^c\}$ \\
    Duals: $(A\bullet B)^c=(A^c \circ \hat{B})$\\
    Boundary extraction: $\beta(A)=A-(A\ominus B)$\\
    Region filling: $X_k=(X_{k-1}\oplus B)\cap A^c\quad k=1,2,3,\ldots$, terminate at $X_k=X_{k-1}$.\\
    Extraction of connected components ($Y$): $X_k=(X_{k-1}\oplus B)\cap Y\quad k=1,2,3,\ldots$\\
    Skeletons: $S(A)=\bigcup_{k=0}^K S_k(A)$, with $S_k(A)=(A \ominus kB)-[(A \ominus kB)\circ B]$, $K=\max\{k\vert (A \ominus kB)\neq \emptyset\}$

    \section{Image Compression}
    Compression ratio: $C_R=\frac{n_1}{n_2}$\\
    Relative data redundancy: $R_D=1-\frac{1}{C_R}$\\
    Coding redundancy: $p_r(r_k)=n_k/n, \quad k=0,1,2,3\ldots,L-1$, $\bar{L}=\sum_{k=0}^{L-1}l(r_k)p_r(r_k)$, $MN\bar{L}$ bits to code $M\times N$ image
    \subsection{Image Compression Systems}
    Mapper (spatial redundancy, reversible), quantizer (psychovisual redundancies, irreversible, not error free), symbol coder\\
    Objective fidelity criteria: $\text{SNR}_\text{ms}=\frac{\sum_{x=0}^{M-1}\sum_{y=0}^{N-1}\hat{f}(x,y)^2}{\sum_{x=0}^{M-1}\sum_{y=0}^{N-1}[\hat{f}(x,y)-f(x,y)]^2}$
    \subsection{Information Theory}
    $I(E)=\log_r \frac{1}{P(E)}=-\log_r P(E)$\\
    Entropy: $H(A)=-\sum_{j=1}^J P(A_j)\log P(a_j)$\\
    For any uniquely decodable code $C$, $H(I)$ is lower bound on average word length $\bar{L}_C$, efficiency $\eta =\frac{H(I)}{\bar{L}_C}$
    \subsection{Error-Free Compression}
    Huffman coding: order top-bottom in decreasing probability, merge bottom two, repeat; Two remaining, assign $0$ and $1$, go back to two probabilities before combination, prefix assigned digit, repeat\\
    Arithmetic coding: Not one-one; Multiply by probability of successive symbol; Check successive range\\
    Bit-plane coding: polynomials, separate each one of 8 bit into individual plane\\
    Run-length coding: Black and white, encode length, row length known; Need specify first run or assume white\\
    Delta compression: Successive differences\\
    \subsection{Lossy Compression}
    Error-free seldom more than $3:1$, lossy can be $>30:1$, virtually indistinguishable at lower ratios,
    principal difference: quantizer block; e.g. lossy predictive coding, optimal quantization, transform coding
\end{multicols}

\end{document}
