%%%%%%%%%%%%%%%%%%%%%%%%%%%%%%%%%%%%%%%%%%%%%%%%%%%%%%%%%%%%%%%%%%%%%%
% writeLaTeX Example: A quick guide to LaTeX
%
% Source: Dave Richeson (divisbyzero.com), Dickinson College
% 
% A one-size-fits-all LaTeX cheat sheet. Kept to two pages, so it 
% can be printed (double-sided) on one piece of paper
% 
% Feel free to distribute this example, but please keep the referral
% to divisbyzero.com
% 
%%%%%%%%%%%%%%%%%%%%%%%%%%%%%%%%%%%%%%%%%%%%%%%%%%%%%%%%%%%%%%%%%%%%%%
% How to use writeLaTeX: 
%
% You edit the source code here on the left, and the preview on the
% right shows you the result within a few seconds.
%
% Bookmark this page and share the URL with your co-authors. They can
% edit at the same time!
%
% You can upload figures, bibliographies, custom classes and
% styles using the files menu.
%
% If you're new to LaTeX, the wikibook is a great place to start:
% http://en.wikibooks.org/wiki/LaTeX
%
%%%%%%%%%%%%%%%%%%%%%%%%%%%%%%%%%%%%%%%%%%%%%%%%%%%%%%%%%%%%%%%%%%%%%%

\documentclass[dvipdfmx,a4paper,10pt,landscape]{article}

\usepackage{amsmath}
\usepackage{stix2}
\usepackage{siunitx}
\usepackage{enumitem}

\usepackage{multicol,multirow}
\usepackage{ifthen}
\usepackage[landscape]{geometry}
\usepackage[colorlinks=true,citecolor=blue,linkcolor=blue]{hyperref}
\setlist[itemize]{noitemsep, nolistsep}

\ifthenelse{\lengthtest { \paperwidth = 11in}}
    { \geometry{top=.5in,left=.5in,right=.5in,bottom=.5in} }
	{\ifthenelse{ \lengthtest{ \paperwidth = 297mm}}
		{\geometry{top=1cm,left=1cm,right=1cm,bottom=1cm} }
		{\geometry{top=1cm,left=1cm,right=1cm,bottom=1cm} }
	}
\pagestyle{empty}
\makeatletter
\renewcommand{\section}{\@startsection{section}{1}{0mm}%
                                {-1ex plus -.5ex minus -.2ex}%
                                {0.5ex plus .2ex}%x
                                {\normalfont\large\bfseries}}
\renewcommand{\subsection}{\@startsection{subsection}{2}{0mm}%
                                {-1explus -.5ex minus -.2ex}%
                                {0.5ex plus .2ex}%
                                {\normalfont\normalsize\bfseries}}
\renewcommand{\subsubsection}{\@startsection{subsubsection}{3}{0mm}%
                                {-1ex plus -.5ex minus -.2ex}%
                                {1ex plus .2ex}%
                                {\normalfont\small\bfseries}}
\makeatother
\setcounter{secnumdepth}{0}
\setlength{\parindent}{0pt}
\setlength{\parskip}{0pt plus 0.5ex}

\DeclareMathOperator{\sinc}{sinc}
\DeclareMathOperator{\rect}{rect}
% -----------------------------------------------------------------------

\begin{document}

\raggedright
\footnotesize

\begin{center}
    \Large{\textbf{EE4704 Cheat Sheet}} \\
\end{center}
\begin{multicols}{3}
    \setlength{\premulticols}{1pt}
    \setlength{\postmulticols}{1pt}
    \setlength{\multicolsep}{1pt}
    \setlength{\columnsep}{2pt}

    \section{Introduction}

    \subsection{Network Delays and Losses}
    $d_{\text{nodal}}=d_{\text{proc}}+d_{\text{queue}}+d_{\text{trans}}+d_{\text{pop}}$\\
    $d_{\text{trans}}=\frac{L\text{(bits)}}{R(\text{bits/sec})}$ ($L$: length of packet, $R$: transmission rate)\\
    $d_{\text{queue}}=d_{\text{trans}}\times l_{\text{queue}}$, ($l_{\text{queue}}$: average length of queue) \\
    $d_{\text{prop}}=\frac{D}{s}$ ($D$: distance between nodes, $s$: propagation speed)\\

    \subsection{Packet Switching vs Circuit Switching}
    \begin{itemize}
        \item great for bursty data
        \item simpler, no call setup
        \item excessive congestion possible: packet delay and loss
        \item protocols needed for reliable data transfer, congestion control
    \end{itemize}

    \subsection{Throughput}
    throughput: rate (bits/time unit) at which bits transferred between
    sender/receiver

    \subsection{Internet Protocol Stack}
    \begin{itemize}
        \item \textbf{application}: supporting network applications
              FTP, SMTP, HTTP
        \item presentation: allow applications to
              interpret meaning of data, e.g., encryption,
              compression, machine-specific
              conventions
        \item session: synchronization, checkpointing,
              recovery of data exchange
        \item \textbf{transport}: process-process data transfer
              TCP, UDP
        \item \textbf{network}: routing of datagrams from source to
              destination
              IP, routing protocols
        \item \textbf{link}: data transfer between neighboring
              network elements
              Ethernet, 802.111 (WiFi), PPP
        \item \textbf{physical}: bits ``on the wire''
    \end{itemize}
    presentation and session from ISO/OSI must be implemented in application \textit{if needed}

    \section{Transport Layer}
    \subsection{Transport Services and Protocols}
    \begin{itemize}
        \item provide logical communication
              between app processes running on
              different hosts
        \item transport protocols run in end
              systems
              \begin{itemize}
                  \item send side: breaks app messages
                        into segments, passes to
                        network layer
                  \item rcv side: reassembles segments
                        into messages, passes to app
                        layer
              \end{itemize}
    \end{itemize}

    \subsection{TCP vs UDP}
    \begin{itemize}
        \item TCP (reliable, in-order delivery)
              \begin{itemize}
                  \item congestion control
                  \item flow control
                  \item connection setup
              \end{itemize}
        \item UDP (unreliable, unordered
              delivery)
              \begin{itemize}
                  \item no-frills extension of
                        ``best-effort" IP
              \end{itemize}
    \end{itemize}
    services not available:
    \begin{itemize}
        \item delay guarantees
        \item bandwidth guarantees
    \end{itemize}

    \subsection{Connection-Oriented Demux}
    TCP socket identified by 4-tuple:
    \begin{itemize}
        \item source IP address
        \item source port number
        \item dest IP address
        \item dest port number
    \end{itemize}

    \subsection{UDP}
    Maybe lost or delivered out-of-order\\
    Connectionless: no handshaking between UDP sender, receiver; each UDP segment handled
    independently of others\\
    UDP use: streaming multimedia
    apps (loss tolerant, rate
    sensitive), DNS, SNMP\\
    Why UDP? No connection delay, simple (stateless), small header size, no congestion control

    \subsection{Pipeling}
    pipelining: sender allows multiple, “in-flight”, yet-to-beacknowledged pkts
    \subsubsection{Go-Back-N}
    \begin{itemize}
        \item sender can have up to N
              unacked packets in pipeline
        \item receiver only sends
              cumulative ack
              \begin{itemize}
                  \item doesn't ack packet if
                        there's a gap
                  \item sender has timer for oldest
                        unacked packet
                        \begin{itemize}
                            \item when timer expires,
                                  retransmit all unacked
                                  packets
                        \end{itemize}
              \end{itemize}
    \end{itemize}

    \subsubsection{Selective Repeat}
    \begin{itemize}
        \item sender can have up to N
              unack'ed packets in pipeline
        \item rcvr sends individual ack for
              each packet
        \item sender maintains timer for each
              unacked packet
              \begin{itemize}
                  \item when timer expires,
                        retransmit only that
                        unacked packet
              \end{itemize}
    \end{itemize}
    for packet n, n should be in [sendbase,sendbase+N] for sender, n can be in [rcvbase,rcvbase+N-1] (buffer or deliver) or [rcvbase-N,rcvbase-1] for receiver, otherwise ignore

    \subsection{TCP}
    ACK number is for next byte expected from other side, cumulative\\
    Handshake: Send SYN, receive SYNACK, send ACK (2nd ACK segment may contain payload)\\
    \subsubsection{TCP Fast Retransmit}
    \begin{itemize}
        \item time-out period often
              relatively long:
              \begin{itemize}
                  \item long delay before
                        resending lost packet
              \end{itemize}
        \item detect lost segments via
              duplicate ACKs.
              \begin{itemize}
                  \item sender often sends many
                        segments back-to-back
                  \item if segment is lost, there
                        will likely be many
                        duplicate ACKs.
              \end{itemize}
    \end{itemize}
    triple duplicate ACKs: resend unacked
    segment with smallest
    seq \#
    \subsubsection{TCP Congestion Control}
    \begin{itemize}
        \item approach:sender increases transmission rate (window size),
              probing for usable bandwidth, until loss occurs
              \begin{itemize}
                  \item additive increase: increase cwnd by 1 MSS every RTT
                        until loss detected
                  \item multiplicative decrease: cut cwnd in half after loss
              \end{itemize}
    \end{itemize}
    \subsubsection{TCP Slow Start}
    \begin{itemize}
        \item when connection begins, increase
              rate exponentially until first loss
              event:
              \begin{itemize}
                  \item initially cwnd = 1 MSS
                  \item double cwnd every RTT
                  \item done by incrementing cwnd
                        for every ACK received
              \end{itemize}
    \end{itemize}
    \subsubsection{TCP: Detecting, Reacting to Loss}
    \begin{itemize}
        \item loss indicated by timeout:
              \begin{itemize}
                  \item cwnd set to 1 MSS;
                  \item window then grows exponentially (as in slow start) to threshold,
                        then grows linearly
              \end{itemize}
        \item loss indicated by 3 duplicate ACKs: TCP RENO
              \begin{itemize}
                  \item dup ACKs indicate network capable of delivering some segments
                  \item cwnd is cut in half window then grows linearly
              \end{itemize}
        \item TCP Tahoe always sets cwnd to 1 (timeout or 3 duplicate acks)
    \end{itemize}

    \subsubsection{TCP: Switching from Slow Start to CA}
    Q: when should the
    exponential increase
    switch to linear?\\
    A: when cwnd gets
    to 1/2 of its value
    before timeout.\\
    Implementation:
    \begin{itemize}
        \item variable ssthresh
        \item on loss event, ssthresh is
              set to 1/2 of cwnd just
              before loss event
    \end{itemize}

    \section{Mobile IP}
    Pros:
    \begin{enumerate}
        \item no need to propagate routing updates again
        \item no need to modify routers
        \item does not create additional entries in routing table
        \item preserves location privacy
    \end{enumerate}
    Cons:
    \begin{enumerate}
        \item inefficiency of triangle routing
        \item packet header overhead of encapsulation
        \item home agent is single point of failure
        \item need to allocate an adress in foreign network
    \end{enumerate}

    \section{Application Layer}
    \subsection{HTTP overview}
    port 80\\
    HTTP response time:
    \begin{itemize}
        \item one RTT to initiate TCP
              connection
        \item one RTT for HTTP request and
              first few bytes of HTTP response
              to return
        \item file transmission time
    \end{itemize}
    HTTP request message:
    \begin{verbatim}
GET /index.html HTTP/1.1\r\n
Host: www-net.cs.umass.edu\r\n
User-Agent: Firefox/3.6.10\r\n
Accept: text/html,application/xhtml+xml\r\n
Accept-Language: en-us,en;q=0.5\r\n
Accept-Encoding: gzip,deflate\r\n
Accept-Charset: ISO-8859-1,utf-8;q=0.7\r\n
Keep-Alive: 115\r\n
Connection: keep-alive\r\n
    \end{verbatim}
    POST method:
    \begin{itemize}
        \item web page often includes form
              input
        \item user input sent from client to
              server in entity body of HTTP
              POST request message
    \end{itemize}

    GET method (for sending data to
    server):
    \begin{itemize}
        \item include user data in URL field of HTTP
              GET request message (following a `?`'):
    \end{itemize}
    HEAD method:
    \begin{itemize}
        \item requests headers (only) that
              would be returned if specified
              URL were requested with an HTTP
              GET method.
    \end{itemize}
    PUT method:
    \begin{itemize}
        \item uploads new file (object) to
              server
        \item  completely replaces file that
              exists at specified URL with
              content in entity body of POST
              HTTP request message
    \end{itemize}

    HTTP response message:
    \begin{verbatim}
HTTP/1.1 200 OK\r\n
Date: Sun, 26 Sep 2010 20:09:20 GMT\r\n
Server: Apache/2.0.52 (CentOS)\r\n
Last-Modified: Tue, 30 Oct 2007 17:00:02
GMT\r\n
ETag: "17dc6-a5c-bf716880"\r\n
Accept-Ranges: bytes\r\n
Content-Length: 2652\r\n
Keep-Alive: timeout=10, max=100\r\n
Connection: Keep-Alive\r\n
Content-Type: text/html; charset=ISO-8859-
1\r\n
\r\n
data data data data data ... 
    \end{verbatim}
    \subsection{Electronic Mail: SMTP}
    \begin{itemize}
        \item uses TCP to reliably transfer
              email message from client to
              server, port 25
              \begin{itemize}
                  \item direct transfer: sending server
                        (acting like client) to receiving
                        server
              \end{itemize}
        \item three phases of transfer
              \begin{itemize}
                  \item SMTP handshaking (greeting)
                  \item SMTP transfer of messages
                  \item SMTP closure
              \end{itemize}
        \item command/response interaction
              (like HTTP)
              \begin{itemize}
                  \item commands: ASCII text
                  \item response: status code and phrase
              \end{itemize}
    \end{itemize}

    \subsection{Mail Access Protocols}
    \begin{itemize}
        \item SMTP: delivery/storage to receiver's server
              mail access protocol: retrieval from server
        \item POP: Post Office Protocol [RFC 1939]: authorization,
              download
        \item Internet Mail Access Protocol [RFC 1730]: more
              features, including manipulation of stored msgs on server
        \item Gmail, Hotmail, Yahoo! Mail, etc.
    \end{itemize}

    \subsection{POP3 (More) and IMAP}
    more about POP3
    \begin{itemize}
        \item previous example uses
              POP3 ``download and
              delete'' mode
              \begin{itemize}
                  \item Bob cannot re-read e-mail
                        if he changes client
              \end{itemize}
        \item POP3 ``download-andkeep'': copies of messages
              on different clients
        \item POP3 is stateless across
              sessions
    \end{itemize}
    IMAP
    \begin{itemize}
        \item keeps all messages in one
              place: at server
        \item allows user to organize
              messages in folders
        \item keeps user state across
              sessions:
              \begin{itemize}
                  \item names of folders and
                        mappings between
                        message IDs and folder
                        name
              \end{itemize}
    \end{itemize}

    \subsection{DNS}
    Domain Name System:\\
    distributed database\\
    implemented in hierarchy of
    many name servers
    application-layer protocol:\\
    hosts, name servers communicate
    to resolve names (address/name
    translation)
    \begin{itemize}
        \item note: core Internet function,
              implemented as applicationlayer protocol
        \item complexity at network's
              ``edge''
    \end{itemize}

    \subsection{DNS Name Resolution}
    iterated query:
    \begin{itemize}
        \item contacted server replies
              with name of server to
              contact
        \item ``I don't know this
              name, but ask this
              server''
    \end{itemize}
    recursive query:
    \begin{itemize}
        \item puts burden of name
              resolution on
              contacted name
              server
        \item heavy load at upper
              levels of hierarchy?
    \end{itemize}

    \section{Multimedia Netowrking}
    \textit{leaky bucket}: limit input to specified burst size and average
    rate
    \begin{itemize}
        \item bucket can hold b tokens
        \item tokens generated at rate r token/sec unless bucket full
        \item over interval of length t: number of packets admitted less
              than or equal to (r t + b)
    \end{itemize}
    Max delay after passing through the bucket: $d_\text{max}=\frac{b_1}{R\cdot w_1/\sum w_j}$

    \section{Ethernet Frame}
    Source IP\\
    Destination IP\\
    Source MAC\\
    Destination MAC

    If not on same LAN, destination MAC is of router

    \section{Security}
    Nonce is used once in a lifetime, used to prevent playback attacks.

    A certificate authority certifies that a public key belongs to a particular
    person or entity. It is useful to prevent adversaries from sending information or
    authenticating while pretending to be someone else.\\

    \section{Network Security}
    \subsection{Principles of Cryptography}
    $m$: plaintext message \\
    $K_A(m)$: ciphertext, encrypted with key $K_A$\\
    $m = K_B(K_A(m))$
    \subsubsection{Public Key Cryptography}
    \begin{itemize}
        \item sender, receiver do not
              share secret key
        \item public encryption key
              known to all
        \item decryption key
              known only to receiver
    \end{itemize}
    \subsubsection{Creating public/private key
        pair}
    \begin{itemize}
        \item choose two large prime numbers p, q.
              (e.g., 1024 bits each)
        \item compute $n = pq$, $z = (p-1)(q-1)$
        \item choose $e$ (with $e<n$) that has no common factors
              with $z$ ($e$, $z$ are
              ``relatively prime'').
        \item choose $d$ such that $ed-1$ is exactly divisible by $z$.
              (in other words: $ed \bmod z = 1$).
        \item public key is $(n,e)$. private key is $(n,d)$.
    \end{itemize}
    to encrypt message m (<n), compute
    $c = m^e \bmod n$\\
    to decrypt received bit pattern, c, compute
    $m = c^d \bmod n$\\
    $m = (m^e \bmod n)^d \bmod n$\\
    useful property: $K_B^-(K_B^+(m))=m=K_B^+(K_B^-(m))$
    \subsection{Message Integrity, Authentication}
    Message Authentication Code (MAC): need to share secret to calculate and verify $H(m||s)$
    \subsubsection{Certification Authorities}
    binds public key to particular
    entity, E
    \subsubsection{end-point authentication}
    nonce: number (R) used only once-in-a-lifetime\\
    playback attack: use nonce and public key cryptography\\
    man in the middle attack
    \subsection{Securing Email}
    random symmetric private key and encrypt it with public key\\
    source authentication, message integrity: digitally sign with private key
    \subsection{SSL/TLS}
    \subsubsection{Almost SSL/TLS}
    3 phases: handshake, key derivation, data transfer\\
    Slice Master Secret key into four:
    \begin{itemize}
        \item EB = session encryption key for data sent from Bob to
              Alice
        \item MB = session MAC key for data sent from Bob to Alice
        \item EA = session encryption key for data sent from Alice to
              Bob
        \item MA = session MAC key for data sent from Alice to Bob
    \end{itemize}
\end{multicols}

\end{document}
